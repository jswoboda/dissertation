\chapter{Incoherent Scatter Radar Signal Model \& Processing}
\label{chapter:isrproc}
\thispagestyle{myheadings}

% set this to the location of the figures for this chapter. it may
% also want to be ../Figures/2_Body/ or something. make sure that
% it has a trailing directory separator (i.e., '/')!
\graphicspath{{2_ISRProc/Figures/}}

%%%%%%%%%%%%%% Intro %%%%%%%%%%%%%%%%%%%%%%%%%%%%%%%%%%%%%

This chapter will give the background on the signal model and processing aspects of ISR. The purpose is to develop the signal model of a pulsed radar systems and then show how that can be extended to the case of incoherent scatter radar. After that the processing that takes place for these systems will be described. 

\section{Incoherent Scatter}


In some radar modalities the system is attempting to measure numerous targets.  The number of targets grows the scatters resemble more of a distribution than a single scatterer.  In the case of ISR the radar is trying to sample the velocity spectrum of the distribution of electrons in the upper atmosphere and ionosphere.  

In ISR the goal of the system is often to sample what is called the ion-line spectrum. The formulation from  \cite{kudeki:milla:1,kudeki:milla:2,Kudeki:2006kx}


\begin{equation}
\label{eq:mainspeceq:body}
\langle \left|n_e(\mathbf{k},\omega)\right|^2\rangle = \frac{|j\omega\epsilon_0 + \sigma_i|^2 \langle |n_{te}(\mathbf{k},\omega)|^2\rangle}{|j\omega\epsilon_0 +\sigma_e+\sigma_i|^2} + \frac{| \sigma_e|^2 \langle |n_{ti}(\mathbf{k},\omega)|^2\rangle}{|j\omega\epsilon_0 +\sigma_e+\sigma_i|^2}.
\end{equation}

\noindent The full derivation of this can be seen in Appendix \ref{appendix1}.
% make ISR spectra


One can see in this formulation that the distribution is actually dependent on the thermal velocity of the ions $\sqrt{K_b(T_e+3T_i)/m_i}$ \cite{chen1984introduction}.  If one multiplies this velocity by the wavenumber $k$ of the radar we actually get a Doppler frequency.  This term is basically a normalization of the frequency space of the distribution to what would be the Doppler of the average thermal speed. The distribution $\langle \left|n_e(\mathbf{k},\omega)\right|^2\rangle $ is basically the distribution of the scatterers at these different speeds.


\section{Radar Signal Processing}
ISR systems like other pulsed radar systems radiate a signal,$y(t)$, that can be represented as a finite length pulse, $s(t)$ modulated by a complex sinusoid,

\begin{equation}
\label{eqn:sigone}
y(t)=s(t)e^{j2\pi f_0 t}
\end{equation}
\noindent where $f_0$ is the transmit frequency in Hz. This transmit frequency can also be equivalently as $ f_0=c/\lambda_0$, where $\lambda_0$ is the wavelength of the transmitted wave and $c$ is the speed of light. This return signal, reflected off of a target, which assuming for now has no motion,

\begin{equation}
\label{eqn:sigone}
y_r(t)=A_0s(t-\Delta t)e^{j2\pi f_0 (t-\Delta t)}
\end{equation}

\noindent where $\Delta T$ round trip time and is $A_0$  is an amplitude of which is amplitude factor including propagation losses and target reflectivity \cite{richards:fundamentalsigproc}. The radar system measures the range $R$, or the distance between the target and sensor, by simply measuring the and using the following conversion,

\begin{equation}
\label{eqn:range_intro}
R=\frac{c\Delta T}{2}.
\end{equation}

\noindent Lastly, the signal is demodulated down to base band and it becomes the following

\begin{equation}
\label{eqn:baseband}
x(t)=A_0s(t-\Delta t)e^{-j2\pi\Delta t}
\end{equation}

Another quantity that can be measured is the line of sight or Doppler velocity. The Doppler for this specific case acts multiplication of the radar signal $s(t)$ with a simple single complex exponential

\begin{equation}
\label{simpledop}
x_d(t) = x(t)e^{j2\pi f_d t},
\end{equation}
 
\noindent where $\omega_d$ is the Doppler frequency of the target. Assuming there are no relativistic effects this frequency can be represented as the following $f_d = 2v/\lambda_0 $, where $v$ is the velocity of the target. 

With ISR a distribution of electrons are actually being probed, so we have more than one target. A better model would be set of scatters each with their weighted return and Doppler frequency, which can be represented as the following

\begin{equation}
\label{multiDop}
\displaystyle x_d(t) = \sum_{n}^{N} x(t)V_ne^{j2\pi f_{n} t}.
\end{equation}

\noindent This then can be extended to a continuum of signals each at each Doppler frequency which becomes

\begin{equation}
\label{conDop}
x_d(t) = \int x(t) V(f)e^{j2\pi ft}.
\end{equation}
\noindent Pulling the $x(t)$ term out of the integral we can see that we are taking the Fourier transform of this relative weighting between each of the scatters and then multiplying it with the signal.  Using simple Fourier properties we can see that this equivalent to a convolution in frequency space of the spectrum of the original radar signal and the Doppler spectrum with the collection of targets.  

The final form of the signal spectrum with Doppler added can be shown as the following

\begin{equation}
\label{finalDop}
x_d(t) = \int \left[\int X(\lambda)V(\lambda-\omega)d\lambda\right] e^{j\omega t}d\omega.
\end{equation}

\noindent This shows that the measured Doppler on the radar signal can be formulated as the convolution of the Fourier transform of radar's signal along with the Doppler spectrum of the target.

\subsection{Applying the Model To Pulse Doppler Radar}

In pulse-Doppler (PD) radar a succession of pulses are sent out modulated by the carrier frequency $f_c$.  Each pulse scatters off of the target and which imparts a Doppler frequency $\omega_d = 2\pi f_c \frac{2v}{c}$, where $v$ is the target velocity and $c$ is the speed of light.  This representation of the Doppler frequency is only valid if the target is non-relativistic.  In the case where we are looking at a single target the return of the $m^{th}$ pulse can be represented in the following way\cite{richards:fundamentalsigproc}

\begin{equation}
\label{pdpulse}
y(t) =  A(t)e^{j\phi}e^{j\omega_dmT},
\end{equation}

\noindent where $T$ is the pulse repetition interval (PRI).  In this case each pulse is sampling the Doppler spectrum at a rate of the pulse repetition frequency (PRF).  Using traditional PD processing the PRF determines the maximum unambiguous Doppler frequency.  For example if one wants a system with a carrier frequency of 10 GHz that will resolve a target going the speed of sound with aliasing in Doppler (approximately 340 m/s) that system must have a PRF greater than 45 kHz if one uses the Nyquist theorem.   

To get the final measurement of this spectrum often a Discrete Fourier transform is applied.  When the data arrives to the radar it is sampled in to specific range gates and pulse samples.  Pulse compression is applied across range to help to localize the signal in range.  This operation is basically applying a filter that is the time reversed conjugate of the base band pulse.  After pulse compression operation Discrete Fourier Transforms are taken across the pulse dimension in each range bin.   The final result is commonly referred to as a range-Doppler map.


%%%%%%%%%%%%%%%%%%%%%%%%%%%%%%%%%%%%%%%%%%%%%%%%%%
\subsection{Applying the model to ISR}

To sample this spectrum one needs a process that can sample this frequency response.  Although the function in Equation \ref{eq:mainspeceq:body} has a number of assumptions built in one could still use it as a way to get a feel for what type of sampling frequencies are required.  If we look at Figure \ref{fig:ispecch2} we can see it seems to have no appreciable content beyond $3\omega_\theta$, thus one needs a sensor that can sample at a frequency of at least $6\omega_\theta$ if we are using the Nyquist theorem.  To give a rough example we can say that one wants to look at hydrogen ions at a  temperature 600 Kelvin with a sensor that has a center frequency of 450 MHz \footnotemark[1].  This will yield an $\omega_\theta/2\pi$ of about 2 kHz and in order to sample that spectrum one would need to sample at about 12 kHz just to get this spectrum.
  
\footnotetext[1]{This is a very simple example and probably not best for the ionosphere.  I probably should use Oxygen ions or some other species for this example.}

If one were to use a pulse-Doppler sort of approach to sampling the process used in the previous example one would need a PRF of about 12KHz.  This PRF would only allow the pulse scatter off of a targets that are no more farther then 125km out.  This would not work for ionosphere measurement when one want to measure out 700km. 

In order to measure this spectrum ISR systems often use an intra-pulse autocorrelation method to measure the Ion-line spectrum.  To do this a pulse with a long time width is sent.  The length is often on the order of a number of range bins.  It is assumed that the plasma from different range's is uncorrelated but since the pulse is longer than a range bin energy scattered from other ranges are summed into other range bins.  Once the correlations are formed a Fourier transform is taken of the autocorrelation functions (ACF), thus yielding a power spectrum for each range.  This operation can also be described in terms of a Wigner-Ville distribution in that we are taking the Fourier transform of a time dependent correlation.\footnotemark[2] This spectrum is again the Doppler spectrum of the distribution of targets though and one is left with a range-Doppler map.

In a sense pulse-Doppler and ISR are attempting to measure the same quality, a Doppler spectrum of some target but they just have different measurement methods.  In PD radar the Doppler spectrum is measured across the pulses while in ISR the Doppler spectrum is measured within the pulse itself.  

This is mainly because of what the different systems are trying to measure.  In most PD systems the required sample rate of the Doppler does not cause high enough PRFs to cause range ambiguities.  Also in detection systems where point targets are being detected range (and Doppler) ambiguities can often be corrected.

In ISR the target being observed is a distribution of scatterers with a fairly large Doppler bandwidth.  The large Doppler bandwidth along with the need to measure parameters at far ranges requires one to develop the Doppler spectrum using information that is available within a pulse.  The pulses themselves are basically used as samples in an averaging of the autocorrelation function to develop a statistically significant representation of the spectrum.
\footnotetext[2]{ I need to work on the wording of this paragraph and add examples}

\begin{figure}[!t]
\centering
\includegraphics[width=6in]{ionlinespec}
\caption{ISR Spectrum.}
\label{fig:ispecch2}
\end{figure}

%%%%%%%%%%%%%% Processing %%%%%%%%%%%%%%%%%%%%%%%%%%%%%%%
\section{ISR Processing}

\subsection{ACF Estimation}
After complex receiver voltage data has been created, it is processed to create estimates of the ACF at desired points of space. This type of processing has been detailed and analyzed by e.g. \cite{farley1969} and other authors \cite{nygren1996}. This processing follows a flow chart seen in Figure \ref{fig:chain}.  Note that we assume here a signal pipeline which creates a single altitude measurement for analysis.  More sophisticated approaches for ISR analysis exist that use information from multiple altitudes, including full profile analysis \cite{RDS:RDS3308}, lag profile inversion \cite{Virtanen:20082vx}, and others, but treatment of these approaches is beyond the scope of this manuscript.

\begin{figure}[!t]
\centering
\includegraphics[width=6in]{datastackchain}
\caption{ISR signal processing chain, with signal processing operations as squares and data products as diamonds.}
\label{fig:chain}
\end{figure}


The lag product formation is an initial estimate of the autocorrelation function. The sampled complex receiver voltage can be represented as $x(n) \in\mathbb{C}^N$ where $N$ is the number of samples in an inter pulse period. For each range gate $m\in 0,1,...M-1$ a complex autocorrelation is estimated for each lag of $l \in 0,1...,L-1$.  To get better statistics this operation is performed for each pulse $j\in 0,1,...J-1$ and then summed over $J$ independent pulses. The entire operation to form the initial estimate of $\widehat{R}(m,l)$ can be seen in Equation \ref{eq:lagpro}:

\begin{equation}
\label{eq:lagpro}
\widehat{R}(m,l) = \displaystyle\sum\limits_{j=0}^{J-1} x(m-\lfloor l/2\rfloor,j)x^*(m+\lceil l/2 \rceil,j).
\end{equation}

The case shown in Equation \ref{eq:lagpro} is a centered lag product.  Other types of lag product calculations are available but generally a centered product is used. In the centered lag product case range gate index $m$ and sample index $n$ can be related by $m=n-\lfloor L/2\rfloor$ and the maximum lag and sample relation is $M=N-\lceil L/2 \rceil$.  This lag product formation is the first step in taking a discrete Wigner Distribution \cite{TFAcohen}. This first step adds a bias to the ACF estimate which acts as a weighting on larger lags, represented as $\mathcal{W}(l)$ where weighting can be calculated from details of the range-lag ambiguity function along the range axis. The expected value for the estimator, assuming the use of a simple uncoded pulse waveform, becomes

\begin{equation}
\label{eq:lagprobias}
\left\langle\widehat{R}(m,l) \right\rangle = \mathcal{W}(l)R(m,l) =\frac{L-l}{L}R(m,l).
\end{equation}
%This specific type of lag product formation is detailed in \cite{farley1969} and had been referred to as unbiased. This terminology does differ from what is used in statistic signal processing literature such as \cite{randomsigshanmugan} where the unbiased autocorrelation function estimate is carried out as so,
%
%\begin{equation}
%\label{eq:lagproub}
%\hat{R}(m,l) = \frac{1}{L-l}\displaystyle\sum\limits_{j=0}^{J-1} x(m-\lfloor l/2\rfloor,j)x^*(m+\lceil l/2 \rceil,j).
%\end{equation}
%
%\noindent With out the $\frac{1}{L-l}$ term the estimator will be windowed with a triangular function thus impacting the estimate of the ISR spectrum as this will act as a convolution in the frequency domain. This bias is taken into account in \cite{farley1969} but it is simply wrapped up into the ambiguity function. 

Applying a summation rule is generally the next step in creating an estimate of the autocorrelation function for single altitude analysis. This is done for a number of reasons, but primarily to improve estimate statistics.  Furthermore, if the right rule is chosen, then the range ambiguity can be made approximately constant across the lags \cite{nygren1996}. The trapezoidal summation rule used in our simulation is a common choice and can be represented as follows

\begin{equation}
\label{eq:sumrule}
\widehat{R}_s(m,l) = \displaystyle\sum\limits_{i=-((v-1)/2+\lceil l/2 \rceil)}^{((v-1)/2+\lfloor l/2\rfloor)} \widehat{R}(m+i,l),
\end{equation}

\noindent where $v$ is the 'volume' index or the number of gates integrated at zero lag (restricted to odd integers here) and $\widehat{R}_s(m,l)$ is the final ACF estimate after the summation rule \cite{nygren1996}. 

% pcom checked for variables, already defined
However, the final result of this summation rule will still lead to a biased ACF. For the uncoded waveform case, this summation rule leads to the ollowing expected value for the estimator \cite{nygren1996},

\begin{equation}
\label{eq:sumruleest}
\left\langle\widehat{R}_s(m,l) \right\rangle  =\frac{v+l}{v\mathcal{W}(0)}\mathcal{W}(l)R(m,l) =\left(-\frac{1}{vL}l^2+\frac{L-v}{Lv}l+1\right)   R(m,l).
\end{equation}
%An example summation rule for a central product is shown in Figure \ref{fig:sumrule}. In the figure the image on the left is a basic representation of an ambiguity function of a long pulse.  Its mirrored on the right with red bars which would show the integration area under it so the ambiguity function for each lag will be of equal size in range. There are a number of different summing rule each with their own trade offs \cite{nygren1996}.

%In the processing this is basically a summing of lags from different ranges. The amount of summing is similar to what is shown in Figure \ref{fig:sumrule}.   



%\begin{equation}
%\label{eq:lagpronoise}
%\hat{R}_w(m,l) = \displaystyle\sum\limits_{j=0}^{J-1} w(m_w-\lfloor l/2\rfloor,j)w^*(m_w+\lceil l/2 \rceil,j),
%\end{equation}

Finally, noise effects are included by subtracting an estimate of the noise correlation from $\widehat{R}_s(m,l)$.  We represent the noise correlation function as $\widehat{R}_w(m,l)$, the ACF estimate of the background noise process of the radar $w(n_w)$ using the steps in Equations \ref{eq:lagpro} and \ref{eq:sumrule}. In a real radar system the noise process is typically sampled either during a calibration period for the radar when nothing is being emitted, or at ranges sufficiently distant that scattered ionospheric signal is assumed to be negligible. For our simulated estimate, we simply use a noise process with the same correlation structure and power as the noise that was added. The final estimate of the autocorrelation function after the noise subtraction and summation rule is represented by $\widehat{R}_f(m,l)$.

%\begin{figure}[!t]
%\centering
%\includegraphics[width=3in]{sumrule}
%% where an .eps filename suffix will be assumed under latex, 
%% and a .pdf suffix will be assumed for pdflatex; or what has been declared
%% via \DeclareGraphicsExtensions.
%\caption{Summation Rule Diagram}
%\label{fig:sumrule}
%\end{figure}

After the final estimation of the spectrum is complete, nonlinear least squares fitting takes place to determine plasma parameters.  
%\pcom{This entire section can be collapsed to one reference and saved for the thesis chapter.}%
%{

\subsection{Parameter Fitting}
The class of nonlinear least squares problems relevant to ISR parameter estimation can be represented as the Êminimization of a cost function of the form \cite{kayvol1},

\begin{equation}
	\mathbf{\hat{p}}= \underset{\mathbf{p}}{\text{argmin}} (\mathbf{y}-\bm{\theta}(\mathbf{p}))^*\bm{\Sigma}^{-1}(\mathbf{y}-\bm{\theta}(\mathbf{p})).
\label{nlls}
\end{equation}

In Equation \ref{nlls}, the data represented as $\mathbf{y}$ would be the final estimate of the ACF $\widehat{R}_f(m,l)$ at a specific range or its spectrum $\widehat{S}_f(m,\omega)$. The parameter vector $\mathbf{P}$ would be the plasma parameters $N_e$, $T_e$, $T_i$ and $V_i$. The fit function, $\bm{\theta}$, is the IS spectrum calculated from models, such as once seen in \cite{kudeki:milla:1}, smeared by the ambiguity function. In the case of the long pulse the ambiguity can be simply applied by multiplying it with the autocorrelation function $R(l)$, if the summation rule is properly applied. In the past the Levenberg-Marquart algorithm to fit data \cite{nikoukar2008}.

%Using the covariance matrix from the fitted parameters, an overall error estimate can be achieved. This matrix is calculated using a numerical approximation to the Jacobian matrix that the function uses to determine the solution. Due to the way the numerical routines solve the problem, this matrix must be multiplied by the error between the estimated parameters and the data,

The last step is to calculate the errors in the parameter estimates. In order to do this a numerical approximation of the jacobian matrix between the data and the ACF, $\mathbf{J}$, at $\mathbf{p}=\mathbf{\hat{p}}$. The formula to estimate the parameter error matrix, $\mathbf{C}_{\mathbf{\hat{p}}}$ according to \cite{Hysell:2000cq}, is


\begin{equation}
\label{eqn:jacinv}
\mathbf{C}_{\mathbf{\hat{p}}}=(\mathbf{J}^T \mathbf{C}^{-1}\mathbf{J})^{-1},
\end{equation}
%\begin{equation}
%\label{eqn:jacinv}
%\bm{\Sigma}_{\mathbf{\hat{p}}}=\frac{(\mathbf{J}^T\mathbf{J})^{-1} (\mathbf{y}-\bm{\theta}(\mathbf{\hat{p}}))^*\bm{\Sigma}^{-1}(\mathbf{y}-\bm{\theta}(\mathbf{\hat{p}}))}{L-N_{\mathbf{p}}},
%\end{equation}

\noindent where $ \mathbf{C}$ is the covariance matrix from the ACFs or spectra depending on what is being fit. The covariance matrix for the ACF is detailed in Equation \ref{eqn:covcalc}, while the covariance matrix of the spectra is simply the ACF matrix but with discrete Fourier Transforms applied to the rows and columns. The variances of the parameters are then taken as the diagonals of the matrix.


%The correlation matrix $\bm{\Sigma}$ is often realized as a diagonal matrix for many ISR systems the variance of the lags or each point of the spectrum being the values. The variance of the ACF estimator can be estimated using the following,
%
%\begin{equation}
%\label{eqn:acfvar}
%\sigma_{\hat{R}(l)}^2=\frac{1}{JL}\displaystyle \sum_{m=-(L-l-1)}^{L-l-1}\left(\frac{L-|m|+1}{L}\right)\left(|\hat{R}(m)|^2 +|\hat{R}(m+l)\hat{R}(m-l)|\right) + \hat{N}^2
%\end{equation}
%
%\noindent where $N$ is the estimated noise power. To estimate the spectrum variance the matrix $\bm{\Sigma}$ is transformed in to the Fourier domain using FFTs (FFT on the columns and IFFT on the rows) so as to model the $\mathbf{F}\bm{\Sigma} \mathbf{F}^*$ matrix operation. 



%%
The diagonal values usually used, noted as $\sigma_i^2$, usually the same unless there is a larger measurement error for one of the lags or spectrums.  The following formula from  \cite{nicollsisrschool2013} can be used:

\begin{equation}
\label{sigpow}
\sigma_i = \frac{S}{\sqrt{J}}\left(1+\frac{1}{SNR}\right).
\end{equation}

\noindent where $S$ is the signal power and $SNR$ is the signal to noise ratio. The noise level can be estimated from the calibration period. 

In the past, ISR researchers have used the Levenberg-Marquart algorithm to fit data \cite{nikoukar2008}. This specific iterative algorithm moves the parameter vector $\mathbf{p}$ by a perturbation $\mathbf{h}$ at each iteration\cite{gavin:2013}. Specifically Levenberg-Marquart was designed to be a sort of meld between two different methods Gradient Decent, and Gauss-Newton. The perturbation vector $\mathbf{h}_{lm}$ can be calculated using the following:

\begin{equation}
\left[ \mathbf{J}^T\bm{\Sigma}^{-1}\mathbf{J}\right]\mathbf{h}_{lm} =\mathbf{J}^T\bm{\Sigma}^{-1}(\mathbf{y}-\bm{\theta}(\mathbf{p}))
\label{hlm}
\end{equation}

\noindent where $\mathbf{J}$ is the Jacobian matrix $\partial \bm{\theta}/\partial \mathbf{p}$ \cite{levenberg1944,marquardt:1963}. 

%Using the covariance matrix from the fitted parameters, an overall error estimate can be achieved. This matrix is calculated using a numerical approximation to the Jacobian matrix that the function uses to determine the solution. The Hessian, $\mathbf{H}$ is then calculated by using the Jacobian and then inverted to get the covariance matrix. Due to the way the numerical routines solve the problem, this matrix must be multiplied by the error between the estimated parameters and the data,
%
%\begin{equation}
%\label{eqn:jacinv}
%\bm{\Sigma}_{\mathbf{\hat{p}}}=\frac{(\mathbf{J}^T\mathbf{J})^{-1} (\mathbf{y}-\bm{\theta}(\mathbf{\hat{p}}))^*\bm{\Sigma}^{-1}(\mathbf{y}-\bm{\theta}(\mathbf{\hat{p}}))}{L-N_{\mathbf{p}}},
%\end{equation}

%\noindent where $N_{\mathbf{p}}$ is the number of parameters being fit. The variances of the parameters are then taken as the diagonals of the matrix.}

 %If the Hessian matrix is undefined so it can not be inverted and a proper estimate of the errors is not possible.

