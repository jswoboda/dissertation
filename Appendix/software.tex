\chapter{Software Packages}
\label{chapter:appsoft}
\thispagestyle{myheadings}

\graphicspath{{Appendix/Figures/}}


This appendix covers the software architecture of three software packages that have been created during the course of research for this thesis. The first package, ISRSpectrum, creates the IS backscatter spectra from a given set of ionospheric plasma state parameters. The next package, called Simulation for Incoherent Scatter Radar (SimISR), can create synthetic ISR data at the level of complex voltages sampled at the antenna. The last major package is called GeoData and is an advanced programing interface (API) that allows researchers to quickly import and incorporate new sensor and information sources. Included are a number of functions to register, interpolate, plot and analyze this data.

\section{ISRSpectrum}
The ISR spectrum code base was developed to create the backscatter spectra from the ionosphere given a set of plasma state parameters. The code is based on the work of \citet{kudeki:milla:1}, which has been discussed in Appendix \ref{appendix1} along with added formulations for multiple ion species.

The software is built using a simple class set up where first the user initializes an object called \textit{ISRSpectrum} using the parameters for the ISR system. After the initial definition a spectrum can be created with the method \textit{getspecsep}, which takes information on the ion species, temperatures, densities. There are optional inputs for magnetic field aspect angle, ion velocities and neutral species to calculate collision frequencies using the methods found in \citet{schunk2004ionospheres}.

\section{SimISR}

SimISR was developed to create synthetic data that one could use to create different processing algorithms and methods while having a known input or "truth" data. This simulator takes plasma fields of plasma parameters, create complex voltages from these parameters and then process that data to the point where it can estimate the input plasma parameters.

The need to create a full 3-dimensional software package was necessitated by the desire to explore the utility of new phased array radar systems and their ability to measure 3-D fields of plasma parameters. This desire to understand the measurement capabilities of electronically scanned ISR led to the following publication, \citet{RDS:RDS20236} where this software package was used to create reconstructions of 3-D fields of plasma parameters.

The software itself has been developed in such a way that the code mirrors the processing, i.e. through the structure of the classes. An object oriented paradigm is used and thus different processes in the processing chain are broken up into different classes, which are: 

\begin{itemize} 
\item IonoContainer - A container class that holds information on the ionosphere or autocorrelation functions (ACFs)/spectra, both intrinsic and estimated.

\item RadarDataFile - A class that holds and operates on the radar data to create estimates of the autocorrelation function. The class takes files containing ISR spectra and then creates ISR data and as a final step outputs instances of the IonoContainer class that holds estimates of the plasma ACF.

\item FitterMethodsGen - A class that applies the fitter to the data and outputs an instance of IonoContainer with the measured parameters. 
\end{itemize}

The overall flow can be seen in Figure \ref{fig:swflow}, where  $\Theta$ is the plasma parameter vector $ g(\Theta)$ is a function that turns the plasma parameters into ISR spectra, $ \mathbf{r}$ is the ACFs/spectra for each point of time and space, $ \mathbf{Lr}$ is the radar's operator on the ACFs/spectra, $ \rho$ is the measured ACFs from the radar and lastly $ \hat{\Theta}$ is the estimate of plasma parameters from $ \rho$ after least squares fitting.

\begin{figure}[h!]
\centering
\includegraphics[width=5.0in]{softwareflowandmath}
\caption{Software flow diagram for SimISR}
\label{fig:swflow}
\end{figure}


%This report is broken up in to the following chapters. The next chapter will cover the method to calculate the ISR spectrum. There are a number of publications on this including \citep{dougherty:farley1960,hagfors1971,sheffield2010} that cover this area but we focus on the treatment found in \citep{kudeki:milla:1}. Next the method to form the ISR data will be shown, this will include the signal processing steps taken to create the data. The processing of the data from estimating the lags to fitting the plasma parameters will then be covered. We will focus on methods for long pulse but we will show where this differentiates when using other waveforms such as alternating codes and Barker codes. Lastly we will show examples of the output of the ISR simulator at a number of different spots in the processing chain.

\section{GeoData}

The GeoData project allows researchers to quickly bring in new sets of data from different sources and analyze it. The API, available both in Python and MATLAB, does this by abstracting the data set into a formatted object which can be manipulated using methods and function that are already included, thus reducing the amount of software that needs to be written. The basic object structure is shown in Figure \ref{fig:objdiag}.

\begin{figure}[h!]
\centering
\includegraphics[width=6.0in]{geodatadiagram}
\caption{The makeup of a GeoData object.}
\label{fig:objdiag}
\end{figure}

The primary variables in a GeoData object are as follows: 
\begin{itemize} 
\item Data from Sensor - This holds the data for the data set. In python this is a dictionary where the keys are the names of the data and the values will be numpy arrays that hold the data. In MATLAB the field names are the data names and the arrays will be the values. Each data set is held in a flattened array structure or can be an NxT array where N is the number of locations of measurements and T will be the number of times.
\item Coordinate System - This string holds the types of coordinates for the data. There is a set number of coordinate types seen in the table below. More can be added as needed.
\item Locations - This will be a NxP array of locations in the coordinate system of choice. P is the number of elements.
\item Sensor Location - This is an array that holds the location of the sensor in wgs84. If there are multiple sensors such as a set of satellite measurements the array will be filled with nans.
\item Times - A Tx2 array of times in POSIX format (i.e. seconds since 1970-01-01 00:00:00 UTC) showing the ending and beginning of a measurement.
\end{itemize}

There are also a number of available coordinate systems that can be used as well. Table \ref{tab:coord} gives the name as used in the code base and a quick description of the system and units.

\begin{table}[]
\centering
\caption{Description of Coordinate Systems.}
\label{tab:coord}
\begin{tabular}{p{1in}p{4in}}
Name       & Description                                                                            \\
wgs84      & Latitude Longitude Altitude (deg,deg,m).                                                \\
Spherical  & Range azimuth and elevation (km, deg, deg) elevation angle is referenced to z=0 plane.  \\
Spherical2 & Range azimuth and elevation (km, deg, deg) elevation angle is referenced to x=y=0 line. \\
ENU        & East north up (m,m,m). sensorloc holds the origin.                                      \\
ECEF       & Earth centered earth fixed (m,m,m).                                                     \\
Cartesian  & Local Cartesian grid (km,km,km). Same as ENU but in km.                
\end{tabular}
\end{table}

The workflow for this package is as follows:
\begin{itemize}
\item Read In Data - The user creates a function that will read the data file into the specific variables for GeoData. Many are already available to the user for common formats.
\item Registration - If using multiple data sets overlap times must be determined. A method named \textit{timeregister} is available to do this automatically.
\item Spatial Registration - This step includes interpolating the data sets into common coordinate systems to allow for plotting.
\item Plotting - There are a number of different plotting tools available including 1-D, 2-D and 3-D plotting methods. These methods are dependent on the coordinate systems that the data are in.
\end{itemize}
