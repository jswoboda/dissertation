\addcontentsline{toc}{chapter}{Curriculum Vitae}

\thispagestyle{empty}

\begin{center}
{\LARGE {\bf CURRICULUM VITAE}}\\
\vspace{0.5in}
{\large {\bf John Swoboda}}
\end{center}

\section*{EDUCATION}{\sl PhD,} Electrical Engineering, January 2017 \\
                      % \sl will be bold italic in New Century Schoolbook (or
	              % any postscript font) and just slanted in
		      %	Computer Modern (default) font
                GPA 3.90/4.00 \\
                Boston University, Boston, MA
	       

{\sl Master of Science,} Electrical Engineering, May 2008 \\
                      % \sl will be bold italic in New Century Schoolbook (or
	              % any postscript font) and just slanted in
		      %	Computer Modern (default) font
                 GPA 3.71/4.00\\
                Thesis Title: Reconstruction of Tomographic Images Corrupted by a Slice Sensitivity Profile With Applications to The Inspection of Manufactured Items\\
 {\sl Bachelor of Science,} Electrical \& Computer Systems Engineering, May 2007  \\
                GPA: 3.74/4.00\\
                Rensselaer Polytechnic Institute, Troy, NY, \\
                
 
 

 
\section*{RESEARCH \\ EXPERIENCE}
 {\sl Research Assistant} \hfill            September 2012 - Present \\
                Electrical Engineering Department, Boston University 
                 \begin{itemize}  \itemsep -2pt %reduce space between items
                 \item Performed data fusion and analysis from varied sensor types for geophysical research.
                 \item Created theoretical frame work for incoherent scatter radar data processing.
                 \item Developed simulations, algorithms for incoherent scatter radar systems.
                 \end{itemize} 

 {\sl Sensor System Engineer} \hfill January 2009 - August 2012\\
                The MITRE Corporation, Bedford MA 
                 	\begin{itemize}  \itemsep -2pt %reduce space between items
                	\item Developed new algorithms for experimental radar systems.
                	\item Created full signal processing level emulations and simulations of radar systems.
                \end{itemize}
 
                {\sl Research Assistant} \hfill            May 2009 - January 2009 \\
                Electrical Engineering Department, Rensselaer Polytechnic Institute 
                 \begin{itemize}  \itemsep -2pt %reduce space between items
                 \item Developed signal processing algorithms applied to image formation radar.
                 \end{itemize} 
                {\sl Research Engineering Intern} \hfill        May 2007 - May 2008\\
                Lickenbrock Technologies, Troy NY
                  \begin{itemize}
                   \item Developed signal processing algorithms for the deblurring of tomographic images.
                   \end{itemize} 
\section*{COMPUTER \\ SKILLS} {\sl Languages \& Software:} Python, MATLAB, C, C$++$, \LaTeX, Git, Microsoft Office.\\
                {\sl Operating Systems:} MacOS, Linux, Windows. 
 
\section*{PUBLICATIONS}   

\begin{itemize}
	\item Swoboda, J., Semeter J., Improvement of Resolution of Incoherent Scatter Radar Using Electronically Scanned Arrays and Inverse Theory, IEEE Symposium on Phased Array Technology, 2016
	\item Swoboda, J., Semeter J., Erickson, P., Zettergren M., Observability of Ionospheric Space-Time Structure with ISR:   A Simulation Study, (Accepted). 
	\item Swoboda, J., J. Semeter, and P. Erickson (2015), Space-time ambiguity functions for electronically scanned ISR applications, Radio Sci., 50, 2015. DOI: 10.1002/2014RS005620

	\item Krishnan, V., Swoboda, J., Yarman, C.E., Yazici, B. , Multistatic Synthetic Aperture Radar Image Formation, IEEE Transactions on Image Processing , vol.19, no.5, pp.1290-1306, May 2010
    \item  Swoboda, J., Yarman, C. E., Yazici, B., Bistatic synthetic aperture radar imaging for arbitrary trajectories in the presence of noise and clutter. Proc. SPIE 7307, Airborne Intelligence, Surveillance, Reconnaissance (ISR) Systems and Applications VI, 73070D April 28, 2009 
\end{itemize}
	
\section*{PRESENTATIONS}
\begin{itemize}
	\item Swoboda, J., Hirsch, M.,Semeter, J., GeoData Python Toolset: High Performance Python for Geoscience, CEDAR Workshop, 2016
    \item Swoboda, J.,  Semeter, J., The "Impulse Response" of Electronically Scanned and Dish Based ISR, URSI, 2016
    \item Swoboda, J., Semeter J., Erickson, P., Zettergren M., Impact of the Forward Model of ESA Based ISR on Measurements from HAARP, CEDAR, 2015
	\item Swoboda, J., Semeter J., Erickson, P., Zettergren M., Three Dimensional Ionosphere Reconstruction from Electronically Steerable ISR, MTSSP, 2015
	\item Swoboda, J., Dahlgren, H., Semeter J., Erickson, P., On the Way to Optimal Processing for Multi-beam RISR Experiments, CEDAR, 2014
    \item Swoboda, J., Semeter J., Erickson, P., Simulation of ISR Data and Application to Spatial Sampling of the Ionosphere, URSI, 2015
\end{itemize}

\section*{POSTER \\ PRESENTATIONS}
\begin{itemize}
\item Swoboda, J., Semeter J., Improvement of Resolution of Incoherent Scatter Radar Using Electronically Scanned Arrays and Inverse Theory, NEROC Symposium, 2016
\item Swoboda, J., Semeter J., Improvement of Resolution of Incoherent Scatter Radar Using Electronically Scanned Arrays and Inverse Theory, IEEE Symposium on Phased Array Technology, 2016
\item Swoboda, J.,  Semeter, J., Resolving Cross Range Gradients in the High Latitude Ionosphere, CEDAR Workshop, 2016
\item Swoboda, J., Hirsch, M.,Semeter, J., GeoData: A Generalized Data Analysis Software Suite, CEDAR Workshop, 2015
\item Swoboda, J., Dahlgren, H., Semeter J., Erickson, P., Plasma Motion Induced Artifacts in 3-D Incoherent Scatter Radar, CEDAR, 2014
\end{itemize}

\section*{CODE / DATASETS}
\begin{itemize}
\item J. Swoboda, M. Hirsch, A. Stuhlmacher, G. Starr, and Semeter, J. (2016). GeoData Python [code]. Zenodo. http://doi.org/10.5281/zenodo.154533
\end{itemize}
