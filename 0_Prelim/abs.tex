% ABSTRACT

Incoherent scatter radar (ISR) systems allow researchers to peer into the ionosphere via remote sensing of intrinsic plasma parameters. ISR sensors have been used since the 1950s and until the past decade were mainly equipped with a single mechanically steerable antenna. As such, the ability to develop a two or three dimensional picture of the plasma parameters in the ionosphere has been constrained by the relatively slow mechanical steering of the antennas. A newer class of systems using electronically steerable array (ESA) antennas have broken the chains of this constraint, allowing researchers to create 3-D reconstructions of plasma parameters. There have been many studies associated with reconstructing 3-D fields of plasma parameters, but there has not been a systematic analysis into the sampling issues that arise. Also, there has not been a systematic study as to how to reconstruct these plasma parameters in an optimum sense as opposed to just using different forms of interpolation.

The research presented here forms a framework that scientists and engineers can use to plan experiments with ESA ISR capabilities and to better analyze the resulting data. This framework attacks the problem of space-time sampling by ESA ISR systems from the point of view of signal processing, simulation and inverse theoretic image reconstruction. We first describe a physics based model of incoherent scatter from the ionospheric plasma, along with processing methods needed to create the plasma parameter measurements. Our approach leads to development of the space-time ambiguity function, forming a theoretical foundation of the forward model for ISR. This forward model is novel in that it takes into account the shape of the antenna beam and scanning method along with integration time to develop the proper statistics for a desired measurement precision.

Once the forward model is developed, we present the simulation method behind the Simulator for ISR (SimISR). SimISR uses input plasma parameters over space and time and creates complex voltage samples in a form similar to that produced by a real ISR system. SimISR allows researchers to evaluate different experiment configurations in order to efficiently and accurately sample specific phenomena. W present example simulations using input conditions derived from a multi-fluid ionosphere model and reconstructions using standard interpolation techniques. Lastly, methods are presented to invert the space-time ambiguity function using techniques from image reconstruction literature. These methods are tested using SimISR to quantify accurate plasma parameter reconstruction over a simulated ionospheric region.
