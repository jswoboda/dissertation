% ABSTRACT

Incoherent scatter radar (ISR) systems can allow researchers to peer into the ionosphere and measure intrinsic plasma parameters. These sensors have been used since the 1950s and have, until recently, been mainly equipped with single dish, mechanically steerable antennas. As such, the ability to develop a two or three dimensional picture of the plasma parameters in the ionosphere has been constrained by the mechanical steering of the antennas. The newer class of systems using electronically steerable arrays (ESA) have broken the chains of this constraint and can allow researchers to create 3-D reconstructions of plasma parameters. There have been many studies associated with reconstructing 3-D fields of plasma parameters but there has not been an analysis into the sampling issues that arise. Also, there has not been a systematic study as to how to reconstruct these plasma parameters in any sort of optimum sense as opposed to just using different forms of interpolation.

The research presented here is a framework that scientists and engineers can use to plan their experiments wth ESA ISR and better analyze the resulting data. This framework attacks the problem of space-time sampling by ESA ISR systems from the point of view of signal processing, simulation and inverse theoretic image reconstruction. The first step in this presentation is giving the signal model of incoherent scatter from the ionosphere along with processing methods needed to create the plasma parameter measurements. This leads to the development of the space-time ambiguity function, which is the theoretical foundation of the forward model for ISR. This forward model takes into account the shape of the antenna beam and scanning method along with the integration time to develop the proper statistics to give precise measurements.

Once the forward model is developed a simulation method behind the Space-Time ISR Simulator (STISRS) is presented. STISRS takes plasma parameters over space and time and can create complex voltage samples as if they came from a real ISR system. It allows researchers to try different experiment set ups in order to sample specific phenomena. For specific examples conditions derived from a multi fluid ionosphere model are used as input and then reconstructed using standard interpolation techniques. Lastly, methods are presented to invert the space-time ambiguity function using techniques from image reconstruction literature. These methods are tested using the STISRS simulator in order to understand the viability of these techniques to accurately reconstruct the field of plasma parameters.
