% This file contains all the necessary setup and commands to create
% the preliminary pages according to the buthesis.sty option.

\title{Space-time Sampling Strategies for Electronically Steerable Incoherent Scatter Radar}

\author{John Swoboda}

% Type of document prepared for this degree:
%   1 = Master of Science thesis,
%   2 = Doctor of Philosophy dissertation.
%   3 = Master of Science thesis and Doctor of Philosophy dissertation.
\degree=2

\prevdegrees{B.S., Rensselaer Polytechnic Institute, 2007 \and M.S.,Rensselaer Polytechnic Institute, 2008} 


\department{Department of Electrical and Computer Engineering}

% Degree year is the year the diploma is expected, and defense year is
% the year the dissertation is written up and defended. Often, these
% will be the same, except for January graduation, when your defense
% will be in the fall of year X, and your graduation will be in
% January of year X+1
\defenseyear{2016}
\degreeyear{2016}

% For each reader, specify appropriate label {First, Second, Third},
% then name, and title. IMPORTANT: The title should be:
%   "Professor of Electrical and Computer Engineering",
% or similar, but it MUST NOT be:
%   Professor, Department of Electrical and Computer Engineering"
% or you will be asked to reprint and get new signatures.
% Warning: If you have more than five readers you are out of luck,
% because it will overflow to a new page. You may try to put part of
% the title in with the name.
\reader{First}{Joshua Semeter, PhD}{Professor of Electrical and Computer Engineering}
\reader{Second}{David Castañón, PhD}{Professor of Electrical and Computer Engineering}
\reader{Third}{S. Hamid Nawab, PhD}{Professor of Electrical and Computer Engineering}
\reader{Fourth}{Philip Erickson, PhD}{Atmospheric Sciences Group MIT Haystack Observatory}

% The Major Professor is the same as the first reader, but must be
% specified again for the abstract page. Up to 4 Major Professors
% (advisors) can be defined. 
\numadvisors=1
\majorprof{Joshua L. Semeter, PhD}{\mbox{Department of Electrical and Computer Engineering}, \mbox{Department of Astronomy}}
%\majorprofb{First M. Last, PhD}{{Professor of computer Science}}
%\majorprofc{First M. Last, PhD}{{Professor of Astronomy}}
%\majorprofd{First M. Last, PhD}{{Professor of Biomedical Engineering}}

%%%%%%%%%%%%%%%%%%%%%%%%%%%%%%%%%%%%%%%%%%%%%%%%%%%%%%%%%%%%%%%%  

%                       PRELIMINARY PAGES
% According to the BU guide the preliminary pages consist of:
% title, copyright (optional), approval,  acknowledgments (opt.),
% abstract, preface (opt.), Table of contents, List of tables (if
% any), List of illustrations (if any). The \tableofcontents,
% \listoffigures, and \listoftables commands can be used in the
% appropriate places. For other things like preface, do it manually
% with something like \newpage\section*{Preface}.

% This is an additional page to print a boxed-in title, author name and
% degree statement so that they are visible through the opening in BU
% covers used for reports. This makes a nicely bound copy. Uncomment only
% if you are printing a hardcopy for such covers. Leave commented out
% when producing PDF for library submission.
%\buecethesistitleboxpage

% Make the titlepage based on the above information.  If you need
% something special and can't use the standard form, you can specify
% the exact text of the titlepage yourself.  Put it in a titlepage
% environment and leave blank lines where you want vertical space.
% The spaces will be adjusted to fill the entire page.
\maketitle
\cleardoublepage

% The copyright page is blank except for the notice at the bottom. You
% must provide your name in capitals.
\copyrightpage
\cleardoublepage

% Now include the approval page based on the readers information
\approvalpage
\cleardoublepage

% Here goes your favorite quote. This page is optional.
\newpage
%\thispagestyle{empty}
\phantom{.}
\vspace{4in}

\begin{singlespace}
\begin{quote}
  \textit{IÕve learned that life is one crushing defeat after another until you just wish Flanders was dead.}\\
  \textit{Marge, this ticket doesn't just give me a seat. It also gives me the right, no, the duty to make a complete ass of myself.}\\
   \textit{The code of the schoolyard, Marge! The rules that teach a boy to be a man. Let's see. Don't tattle. Always make fun of those different from you. Never say anything, unless you're sure everyone feels exactly the same way you do.}\\
   \textit{To alcohol! The cause of, and solution to, all of life's problems.}\\
   \textit{Old people don't need companionship. They need to be isolated and studied so it can be determined what nutrients they have that might be extracted for our personal use.}\\
  \hfill{Homer J. Simpson}
  %\textit{Noctes atque dies patet atri janua Ditis;}\\*
  %\textit{Sed revocare gradum, superasque evadere ad auras,}\\
  %\textit{Hoc opus, hic labor est.}\hfill{Virgil (from Don's thesis!)}
\end{quote}
\end{singlespace}

% \vspace{0.7in}
%
% \noindent
% [The descent to Avernus is easy; the gate of Pluto stands open night
% and day; but to retrace one's steps and return to the upper air, that
% is the toil, that the difficulty.]

\cleardoublepage

% The acknowledgment page should go here. Use something like
% \newpage\section*{Acknowledgments} followed by your text.
\newpage
\section*{\centerline{Acknowledgments}}
There's an old saying, "it takes a village to raise a child." A similar statement can be made for a PhD student and their thesis. The first part is in this village that must be mentioned is my committee starting with my advisor, Professor Josh Semeter, who has helped me chase my ideas to fruition and create a piece of my own scholarship. Dr. Phil Erickson at MIT Haystack Observatory has been incredibly helpful and patient in developing this work and without his guidance this would not be possible. Both Professor Hamid Nawab and Professor David Castañón have been very helpful through the great classes they have taught and lending advice on how best to tackle problems.

Before more specific people are detailed I think it needs to be said that Boston University has some incredible institutions within it. The two that have had the most impact on have been ECE department and the Center for Space Physics. These two institutions have exposed me to a number of different of fascinating ideas and people that I would be hard pressed to find anywhere else.

I have had also a large amount of help from other researchers in the geospace field. This includes Dr. Hanna Dahlgren who was hugely helpful when I was first getting started in the lab. Dr. Matt Zettergren has been extremely helpful in providing simulation data and invaluable advice on other areas.

The are a number of other staff members I would like to thank from MIT Haystack Observatory for all there help. This includes Anthea Coster, Frank Lind, Victor Pankratius, Bill Rideout and Juha Vierinen. 

Staff members from SRI international, have been very helpful as well. This include Mike Nicolls, Steven Chen, Roger Varney and Mary Mc Cready.
 
Accompanying me on my wild ride through academia are my labmates, Nithin, Brent, Michael, Chhavi, Hassan, Sebastian, Greg and Thomas. They have all been excellent collaborators and friends. I also have to add into this mix Matt and Dustin, who although are in a different department, need to be mentioned as if they were part of this group.



\vskip 1in

\noindent

\cleardoublepage

% The abstractpage environment sets up everything on the page except
% the text itself.  The title and other header material are put at the
% top of the page, and the supervisors are listed at the bottom.  A
% new page is begun both before and after.  Of course, an abstract may
% be more than one page itself.  If you need more control over the
% format of the page, you can use the abstract environment, which puts
% the word "Abstract" at the beginning and single spaces its text.

\begin{abstractpage}
% ABSTRACT

Have you ever wondered why this is called an \emph{abstract}? Weird thing is
that its legal to cite the abstract of a dissertation alone, apart from the
rest of the manuscript.

\end{abstractpage}
\cleardoublepage

% Now you can include a preface. Again, use something like
% \newpage\section*{Preface} followed by your text

% Table of contents comes after preface
\tableofcontents
\cleardoublepage

% If you do not have tables, comment out the following lines
\newpage
\listoftables
\cleardoublepage

% If you have figures, uncomment the following line
\newpage
\listoffigures
\cleardoublepage

% List of Abbrevs is NOT optional (Martha Wellman likes all abbrevs listed)
\chapter*{List of Abbreviations}
\begin{center}
  \begin{tabular}{lll}
    \hspace*{2em} & \hspace*{1in} & \hspace*{4.5in} \\
    ISR  & \dotfill & Incoherent Scatter Radar \\
    IS & \dotfill & Incoherent Scatter\\
    AMISR  & \dotfill & Advance Modular Incoherent Scatter Radar \\
    PFISR & \dotfill & Poker Flat Incoherent Scatter Radar\\
    RISR & \dotfill & Resolute Bay Incoherent Scatter Radar\\
    SuperDARN & \dotfill & Super Dual Auroral Radar Network\\
    ACF & \dotfill & Autocorrelation Function\\
    PRF & \dotfill & Pulse Repetition Frequency \\
    IPP & \dotfill & Interpulse Period\\
    ESA  & \dotfill & Electronically Steerable Array \\
    PBI & \dotfill & Poleward Boundary Intensification \\
    SimISR & \dotfill & Simulator for Incoherent Scatter Radar \\
    DFT & \dotfill & Discrete Fourier Transform \\
    MTI & \dotfill & Moving Target Indicator \\
    I\textbackslash Q & \dotfill & Inphase and Quadrature\\
    SNR & \dotfill & Signal to Noise Ratio\\
    CWGN & \dotfill & Complex White Gaussian Noise\\
    RMSE & \dotfill & Root Mean Squared Error\\
    PRI & \dotfill & Pulse Repetition Interval\\
    HAARP & \dotfill & High Frequency Active Auroral Research Program\\
    API & \dotfill & Advanced Programing Interface

 
  \end{tabular}
\end{center}
\cleardoublepage

% END OF THE PRELIMINARY PAGES

\newpage
\endofprelim
