\chapter{ISR Inversion}
\label{chapter:inversion}
\thispagestyle{myheadings}

% set this to the location of the figures for this chapter. it may
% also want to be ../Figures/2_Body/ or something. make sure that
% it has a trailing directory separator (i.e., '/')!
\graphicspath{{5_Inversions/Figures/}}

%%%%%%%%%%%%%%%%%%%%%%%%%%%%%%%%%%%%%%%%%%%%%%%%%%%%%%%%%%%%%%%%%%%%%%%%%
This chapter will describe methods developed to invert the ISR space-time ambiguity function and improve on current reconstruction schemes. The first section will describe these inversion methods which are different types of constrained least squares. The next section will show examples using runs from STISRS to determine the utility of these inversion on reconstructing the true plasma parameters.

\section{Method Description}

As discussed in Section \ref{sec:imgrec}, the main trade-off between the two approaches is computational complexity versus accuracy of solution. Full profile analysis and other parametric regularization schemes require a large amount of computation to find a solution as the operator between the plasma parameters and ISR spectra is non-linear and thus various techniques from are not useful. With data-based regularization techniques the computational requirements are significantly lowered as on is performing a linear inversion. Still this can lead to estimated ACFs that can not be created by the incoherent scattering operator.

In order to reconstruct the field of plasma parameter values we treat the space-time ambiguity as a linear operator. This allows for the use of inversion schemes from image processing. In this case objective function will be 

\begin{equation}
\label{eqn:lstsqrs:opt}
\argmin\limits_{R} \| y_s(\tau_s,\mathbf{r}_s,t_s)- \int L(\tau_s,\mathbf{r}_s,t_s,\tau,\mathbf{r},t)R(\tau,\mathbf{r},t)dVdtd\tau \|^2.
\end{equation}

\noindent If Equation \ref{eqn:lstsqrs:opt} is discretized and the data and input lags are rasterized into a vector format it becomes 

\begin{equation}
\label{eqn:lstsqrs:disc}
\argmin\limits_{\mathbf{r}} \| \mathbf{y_s} -\mathbf{ L}\mathbf{r} \|^2.
\end{equation}
\section{Simulation Examples}