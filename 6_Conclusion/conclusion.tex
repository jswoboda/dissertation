\chapter{Conclusions and Future Work}
\label{chapter:Conclusions}
\thispagestyle{myheadings}

% set this to the location of the figures for this chapter. it may
% also want to be ../Figures/2_Body/ or something. make sure that
% it has a trailing directory separator (i.e., '/')!
\graphicspath{{6_Conclusion/Figures/}}
This chapter contains conclusions, discussions and a summery of this dissertation. Part of the discussions will outline future work from the results of the thesis.

This thesis showed the outlined some of the basic theory behind ISR. It then followed with the modeling of the space-time ambiguity function which gives a forward model for ISR systems with electronically scanned arrays. A methodology to simulate complex voltage data for ISR was shown and is available as a software package called SimISR. Lastly using the space-time ambiguity a method to invert ISR data within the frame of reference of moving plasma and improve the resolution of the data has been developed. The inversion method was tested using SimISR in from a set of two-dimensional ionosphere plasma state parameters.

\section{Space-Time Ambiguity Function}
The space-time ambuigty function 
Section \ref{sec:sptimesamp} Section \ref{sec:sptimeamb} Section \ref{sec:frametrans} 
\section{SimISR}
Section \ref{sec:simex} Section \ref{sec:simmeth} Section \ref{sec:simex}
\section{Inversion of ISR Data}
Section \ref{sec:isrlit} Section \ref{sec:isralg} Section \ref{sec:results}
\section{Future Research Directions}

\subsection{Experiment Planning Using SimISR}

\subsection{Training Sets for Parameter Fits}

\subsection{Inversion Techniques}
