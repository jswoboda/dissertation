\chapter{Conclusions and Future Work}
\label{chapter:Conclusions}
\thispagestyle{myheadings}

% set this to the location of the figures for this chapter. it may
% also want to be ../Figures/2_Body/ or something. make sure that
% it has a trailing directory separator (i.e., '/')!
\graphicspath{{6_Conclusion/Figures/}}
This chapter contains conclusions, discussions and a summery of this dissertation. Part of the discussions will outline future work from the results of the thesis.

This thesis showed the outlined some of the basic theory behind ISR. It then followed with the modeling of the space-time ambiguity function which gives a forward model for ISR systems with electronically scanned arrays. A methodology to simulate complex voltage data for ISR was shown and is available as a software package called SimISR. Lastly using the space-time ambiguity a method to invert ISR data within the frame of reference of moving plasma and improve the resolution of the data has been developed. The inversion method was tested using SimISR in from a set of two-dimensional ionosphere plasma state parameters.

\section{Space-Time Ambiguity Function}
The space-time ambuigty function 
Section \ref{sec:sptimesamp} Section \ref{sec:sptimeamb} Section \ref{sec:frametrans} 
This publication has laid the foundation for the optimal analysis of volumetric data acquired from electronically steerable ISR systems. The framework developed here takes into account the full antenna beam pattern, pulse pattern and time integration. Through simulations, we have shown how plasma motion can impact reconstruction of parameters which, compounded with the non-linear nature of the parameter fitting step, can create errors which are potentially unexpected and hard to predict. Lastly, we briefly outlined a number of possible mitigation approaches improving measurements derived from ESA ISRs.

\section{SimISR}
Section \ref{sec:simex} Section \ref{sec:simmeth} Section \ref{sec:simex}

We have constructed SimISR, a simulation tool encapsulating the full ISR measurement process, incorporating ESA radar capabilities and the full radar space-time ambiguity along with inherent ISR error sources. Possible uses for SimISR in the research community have also been discussed and examples have been shown. These examples show how one can use the simulator to create large statistical data sets and also to more optimally design ionospheric radar experiments in the ISR space within the inherently large number of free parameters afforded by the radar control parameters. 

In the future, SimISR development will continue by adding new radar waveform modes, as currently only Barker code and uncoded single pulse modulations are available at this time. The simulator can also be used to create synthetic data for traditional single antenna based system design applications. Other possible expansions of the simulator include capability to calculate returns from each receiver element in a ESA based ISR system, such as the planned architecture of EISCAT-3D, along with multi-static radar capabilities. These future additions will increase the simulator's value to designers who wish to more optimally exploit the capabilities of new systems. 

A more immediate application of the SimISR tool can aid researchers in experiment planning. There are a number of phenomena that change on very small spatio-temporal scales, e.g. at high latitudes, and capturing observations of them would greatly benefit from optimization of the experiment set up. Researchers can use SimISR to iterate through different set ups for their experiments, as opposed to a heuristic selection of a single observational approach where prior use is the sole design factor. The ability of SimISR to directly create complex receiver voltage data provides a significant and novel capability, as some geophysical phenomena, such as those that occur at time scales on the order of an IPP, can only be fully explored at this data level. This could lead to researchers coming up with new ways to analyze data, such as novel integration schemes to resolve phenomena at small spatio-temporal scales.

\section{Inversion of ISR Data}
Section \ref{sec:isrlit} Section \ref{sec:isralg} Section \ref{sec:results}
\section{Future Research Directions}

\subsection{Experiment Planning Using SimISR}

Researchers have used ISR as a method to measure electron energy spectra \cite{Semeter:2005foa}. In lower altitudes the electron density measurements can be used the estimate production rates, which can be inverted to show the energy spectra of photons impacting the ionosphere and give remote information about magnetospheric activity. Using SimISR researchers could perform simulations to determine the limits of these techniques and plan experiments where they can be used to great affect.

There are many different types of phenomena in the ionosphere which can be spread across numerous spatial and time scales. There are many examples of this in the aurora Ionosphere but SimISR can create complex voltages. These include pulsating aurora which can change on times scales of a second ha 


\subsection{Training Sets for Parameter Fits}

\subsection{Inversion Techniques}
